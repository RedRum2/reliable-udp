\subsection{Stratificazione del sistema}
Per realizzare un'applicazione basata su UDP che garantisca una comunicazione 
affidabile è stato necessario introdurre a livello applicativo uno strato che
si ponesse logicamente tra applicazione e livello di trasporto e che astraesse
un protocollo di trasporto affidabile.\\
Tale strato di ``pseudo-trasporto'', (o livello di trasporto virtuale), fornisce un'interfaccia all'applicazione 
per mezzo delle funzioni \emph{rdt\_send} e \emph{rdt\_recv} simile a quella 
offerta dalle API \emph{sendto} e \emph{recvfrom} utilizzate per UDP, 
quindi tutto ciò che riguarda l'implementazione necessaria a garantire 
affidabilità è trasparente al livello applicativo puro.\\
Tuttavia la trasparenza non è del tutto completa perché, affinché questo strato
di trasporto virtuale funzioni, è necessario inizializzare le strutture che ne 
fanno parte, sia da un lato che dall'altro della comunicazione con gli stessi 
parametri. Questo implica che prima di poter comunicare in modo affidabile, 
client e server devono accordarsi sulla scelta dei parametri, in particolare, 
vengono impostati all'avvio del server, per poi essere inviati ai client che 
vogliono essere serviti.\\
Un'altra caratteristica utile, è che utilizzare la \emph{rdt\_recv}, è 
funzionalmente identico ad effettuare una \emph{read} da una socket in TCP, 
infatti è possibile effettuare letture successive dello stesso messaggio, senza 
che gli altri dati che si devono ancora leggere vadano persi. 
