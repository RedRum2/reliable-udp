Per avviare l'installazione del sistema basta semplicemente eseguire il file
\emph{install.sh} che farà partire la compilazione del codice sorgente e 
creerà una cartella \emph{server} ed una \emph{client} contenenti i
relativi file binari, inoltre verranno copiati alcuni file di esempio nella
cartella del server per poter testare subito l'applicazione.\\
Se il procedimento non dovesse funzionare provare a concedere i permessi di
esecuzione al file di installazione tramite il comando:

\emph{chmod -x install.sh}\\
oppure provare ad eseguire il file all'interno del terminale:

\emph{./install.sh}
