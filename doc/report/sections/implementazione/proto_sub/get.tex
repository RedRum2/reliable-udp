\subsubsection{Comando GET}
Nel messaggio di comando get, i primi 8 bit che specificano il comando sono seguiti dal nome del file che si vuole scaricare, questo campo è composto da un numero indefinito di byte, il server lo interpreta come una stringa, pertanto legge fintanto che non trova il terminatore di stringa.

%	+-------+-----------------------+
%	|  GET  |	   file_name     	|
%	+-------+-----------------------+
%  	  8 bit

Il server una volta ottenuto il nome del file, controlla se è presente in memoria e, in caso affermativo, prepara il messaggio di risposta così strutturato:
i primi 8 bit contengono una costante che indica che il file esiste, i successivi 64 bit la dimensione del file, infine segue l'intero file.

%+---------+-------------------------+---------------------------+
%| GET_OK  |	       file_size        |	..		file		..	|
%+---------+-------------------------+---------------------------+
%  8 bit				64 bit
	
Se il file non esiste, viene inviato un messaggio di soli 8 bit contente la costante che ne indica l'assenza.

%+-----------+
%| GET_NOENT |
%+-----------+
%  	8 bit
