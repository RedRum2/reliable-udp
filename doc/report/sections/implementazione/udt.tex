\subsection{Simulazione rete inaffidabile}
Il progetto è stato testato ed eseguito all'interno di una rete locale, pertanto è stato necessario simulare la perdita dei pacchetti.\\
Ciò è stato fatto introducendo delle opportune funzioni per l'invio dei dati, le quali inviano se e soltanto se viene estratto un numero casuale maggiore di una data probabilità di perdita.

\begin{lstlisting}[title=simul\_udt.c]
ssize_t udt_sendto(int sockfd, const void *buf, size_t len, 
				 const struct sockaddr *addr, socklen_t addrlen,
				 double loss)
{
	ssize_t retval = len;

	// necessary flow control into a local network
	if (loss < 0.1)
		usleep(50);

	if (randgen() > loss) 
		retval = sendto(sockfd, buf, len, 0, addr, addrlen);

	return retval;
}


ssize_t udt_send(int sockfd, void *buf, size_t size, double loss)
{
	return udt_sendto(sockfd, buf, size, NULL, 0, loss);
}


double randgen(void)
{
	static bool seed = false;
	if (!seed) {
		srand48(time(NULL));
		seed = true;
	}
	return drand48();
}
\end{lstlisting}
