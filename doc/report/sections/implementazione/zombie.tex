\subsection{Gestione dei processi}
Una questione non trascurabile è sicuramente quella legata alla gestione
dei processi del sistema.\\
Essendo il server un modello multiprocesso, è fondamentale che ci sia un 
giusto controllo sui processi, in particolare è necassario che ogni 
processo termini la propria esecuzione se il client di cui si occupa non
effettua richieste.\\
Affinché ciò si verifichi, il servizio di ricezione rimane bloccato in 
lettura sulla socket per un tempo limitato, infatti se non riceve segmenti
in questo arco di tempo, viene chiamata la \emph{exit()} ed il processo 
termina. In tal caso la connessione viene considerata scaduta.
\begin{lstlisting}[title=transport.c]
void *recv_service()
{
    .....

    /* set connection timeout */
    timeout.tv_sec = 60;
    timeout.tv_usec = 0;
    if (setsockopt(sockfd, SOL_SOCKET, SO_RCVTIMEO, &timeout, 
        sizeof(timeout)) == -1)
        handle_error("recv_service - setting socket timeout");

    for (;;) {

        r = read(sockfd, buffer, max_recvsize);
        if (r == -1) {
            if (errno == EINTR)
                // signal interruption
                continue;
            if (errno == EAGAIN || errno == EWOULDBLOCK) {
                // timeout expired: close connection
                puts("Connection expired");
                exit(EXIT_SUCCESS);
            }
            handle_error("recv_service - read()");
        }

        /* segment received */
        .....

        /* ACK received */
        .....
    }
    .....
}
\end{lstlisting}
Quando un processo server termina la propria esecuzione, invia il segnale
SIGCHILD al processo server principale, che viene subito intercettato in 
modo tale che possano essere liberate le risorse.\\
Prima della creazione dei processi figli, il processo padre effettua la
la registrazione di un handler per il segnale SIGCHILD, il quale 
chiama la funzione  \emph{wait\_pid} in maniera non bloccante soltanto quando 
se ne ha effettivamente bisogno.
\begin{lstlisting}[title=server.c]
int main()
{
    ..... 

    /* bind the socket to the address */
    if (bind(sockfd, (struct sockaddr *) &servaddr, 
             sizeof(servaddr)) == -1)
        handle_error("bind()");

    /* register SIGCHLD signal handler */
    register_zombie_handler();

    for (;;) {
        /* wait for connection requests */
        /* create a new proccess to handle the client requests */
        ..... 
    }
    .....
}

void register_zombie_handler(void)
{
    struct sigaction sa;

    sa.sa_handler = sig_zombie_handler;
    sa.sa_flags = SA_RESTART;
    sigemptyset(&sa.sa_mask);

    if (sigaction(SIGCHLD, &sa, NULL) == -1)
        handle_error("sigaction()");
}

void sig_zombie_handler(int sig)
{
    pid_t pid;
    (void) sig;

    while ((pid = waitpid(-1, NULL, WNOHANG)) > 0)
        ;

    return;
}
\end{lstlisting}
