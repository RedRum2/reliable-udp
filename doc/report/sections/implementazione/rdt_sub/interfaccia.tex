\subsubsection{Interfaccia}
L'interfaccia che offre questo stato di trasporto virtuale, come descritto nei primi capitoli, è simile a quella che un programmatore ha a disposizione per un protocollo TCP.\\
Per richiamare la funzione per l'invio dei dati (\emph{rdt\_send}) basta specificare l'indirizzo e la dimensione del buffer contente i dati da inviare, mentre per la ricezione (\emph{rdt\_recv}) l'indirizzo e la dimensione del buffer che riceverà i dati.\\
A differenza delle comuni \emph{read} e \emph{write} non va specificato il file descriptor della socket, perché il servizio di trasporto virtuale si basa su una connessione, per cui la socket è fissata al momento dell'instaurazione della connessione, che avviene quando viene chiamata la funzione \emph{init\_transport}.\\
Inoltre la funzione \emph{rdt\_read\_string} permette di leggere una stringa di una certa lunghezza massima dal messaggio arrivato, cosa necessaria per poter leggere il nome del file che si vuole scaricare.

\begin{lstlisting}[title=transport.h].
void init_transport(int sockfd, struct proto_params *params);
void rdt_send(const void *buf, size_t len);
void rdt_recv(void *buf, size_t len);
ssize_t rdt_read_string(char *buf, size_t size);
\end{lstlisting}
