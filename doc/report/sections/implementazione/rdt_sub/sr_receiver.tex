\subsubsection{Selective Repeat: ricevente}
Il lato mittente del protocollo viene svolto soltanto dal 
\emph{recv\_service}, che si occupa di gestire il riordino
e la consegna a livello applicativo dei segmenti che giungono 
dalla rete, ed eventualmente di inviare i riscontri al mittente.
\begin{lstlisting}[title=transport.c]
void *recv_service(void *p)
{
    .....

    for (;;) {

        r = read(sockfd, buffer, max_recvsize);

        if (r == -1) {
            // error ot timeout .....     
        }

        /* segment received */
        if (r == sizeof(struct segment)) {

            sgt = (struct segment *) buffer;
            if (process_segment(
                sgt, segments_cb, &recv_window, cb)) {
                /* send ACK */
                if (udt_send
                    (tools->sockfd, &sgt->seqnum, sizeof(uint8_t),
                     loss) == -1)
                    handle_error("udt_send() - sending ACK");
            }
            continue;
        }

        /* ACK received */
        if (r == sizeof(acknum)) {
            // ACK work .....
            continue;
        }
    }
    .....
}
\end{lstlisting}
Il thread, bloccato in lettura sulla socket, quando riceve un segmento
chiama la funzione \emph{process\_segment} che
\begin{lstlisting}[title=transport.c]
bool process_segment(struct segment *sgt, struct segment *segments_cb,
                     struct window *w, struct circular_buffer *cb)
{
    static unsigned int S = 0;
    unsigned int i, s, seqnum;

    seqnum = sgt->seqnum;
    fprintf(stderr, "received segment %u\n", seqnum);

    if (in_window(w, seqnum)) {

        /* calculate seqnum distance from the base of the window */
        i = distance(w, seqnum);

        /* check if the segment is a duplicate */
        if (is_duplicate(w, i)) {
            //fputs("Already received\n", stderr);
            return true;
        }

        /* store the segment */
        segments_cb[(S + i) % w->width] = *sgt;

        /* mark segment as arrived */
        if (set_bit(&w->ack_bar, i) == -1)
            handle_error("set_bit()");
        fprint_window(stderr, w);

        if (seqnum == w->base) {
            /* calculate the number of consecutive segments */
            s = calc_shift(w);
            /* deliver consecutive segments */
            for (i = 0; i < s; i++) {
                deliver_segment(cb, segments_cb + S);
                S = (S + 1) % w->width;
            }
            /* update window indexes */
            shift_window(w, s);
            w->base = (w->base + s) % MAXSEQNUM;
        }
        return true;
    } else if (in_prewindow(w, seqnum)) {
        //fputs("Already received\n", stderr);
        return true;
    }
    return false;
}
\end{lstlisting}

\begin{lstlisting}[title=transport.c]
void deliver_segment(struct circular_buffer *cb, struct segment *sgt)
{
    if (pthread_mutex_lock(&cb->mtx) != 0)
        handle_error("pthread_mutex_lock");

    /* check free space */
    while (space_available(cb->S, cb->E, CBUF_SIZE) <= MSS) 
        if (pthread_cond_wait(&cb->cnd_not_full, &cb->mtx) != 0)
            handle_error("pthread_cond_wait");

    memcpy_tocb(cb->buf, sgt->payload, sgt->size, cb->E, CBUF_SIZE);
    cb->E = (cb->E + sgt->size) % CBUF_SIZE;

    if (pthread_cond_signal(&cb->cnd_not_empty) != 0)
        handle_error("pthread_cond_signal");
    if (pthread_mutex_unlock(&cb->mtx) != 0)
        handle_error("pthread_mutex_unlock");
}
\end{lstlisting}
