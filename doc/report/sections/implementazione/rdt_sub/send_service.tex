Il sistema di attesa è stato implementato tramite un meccanismo di segnalazione
di eventi basato su variabili di condizione.\\
Il thread infatti attende fintanto che non viene segnalata una condizione di
evento, dopodiché esso si sveglia ed in base al tipo di evento 
esegue il compito associato.\\
La struttura \emph{event} è composta dalle due variabili condizione 
\emph{cond\_event} e \emph{cond\_no\_event}, che indicano rispettivamento
il verificarsi di un evento ed il caso opposto, ovvero che non vi è un evento
da gestire, l'intero \emph{type} invece specifica il tipo di evento che si è
verificato e l'intero a 8 bit \emph{acknum} contiene il numero di sequenza del
segmento per il quale si è ricevuto un ack.

\begin{lstlisting}[title=event.h]
#define NO_EVENT	0
#define PKT_EVENT	1
#define ACK_EVENT	2

struct event {
	pthread_mutex_t mtx;
	pthread_cond_t cnd_event;
	pthread_cond_t cnd_no_event;
	unsigned int type;
	uint8_t acknum;
};

int cond_event_signal(struct event *e, unsigned int event_type);
int cond_ack_event_signal(struct event *e, uint8_t acknum);
\end{lstlisting}

La funzione \emph{cond\_event\_signal} permette di inviare un segnale che indica il
verificarsi della condizione \emph{cond\_event} relativa ad uno specifico tipo di 
evento.\\
La funzione \emph{cond\_ack\_event} è relativa soltanto all'avento di ricezione
di un ack, e permette, oltre che di segnalare l'evento, anche di specificare
il numero di sequenza del segmento per il quale si è ricevuto l'ack.

In questo modo vengono segnalati gli eventi di consegna di dati dall'applicazione
e di arrivo di un ack. La scadenza di un timeout invece avviene semplicemente 
impostando un tempo limite di attesa per la funzione \emph{pthread\_cond\_timedwait},
al termine del quale il thread si risveglia e verrà restituito il valore ETIMEDOUT 
che indica tale evento.

\begin{lstlisting}[title=transport.c]
void *send_service(void *p)
{
		.......

    if (pthread_mutex_lock(&e->mtx) != 0)
        handle_error("pthread_mutex_lock");

    for (;;) {

        e->type = NO_EVENT;
        if (pthread_cond_broadcast(&e->cnd_no_event) != 0)
            handle_error("pthread_cond_broadcast()");

        condret = 0;
        while (e->type == NO_EVENT && condret != ETIMEDOUT) {
            // no events and timeout not expired

		.......

            condret = pthread_cond_timedwait(&e->cnd_event,
                      &e->mtx, &wait_time);
            if (condret != 0 && condret != ETIMEDOUT)
                handle_error("pthread_cond_timedwait");
        }

        /* TIMEOUT EVENT */
        if (condret == ETIMEDOUT) {
            // timeout work .......
            continue;
        }

        switch (e->type) {

        case PKT_EVENT:
            // pkt work .......
            break;

        case ACK_EVENT:
            // ack work .......
            break;
        }
    }

    if (pthread_mutex_unlock(&e->mtx) != 0)
        handle_error("pthread_mutex_unlock");

		.......

}
\end{lstlisting}

L'accesso esclusivo alla variabile \emph{event} è garantito dalla presenza
del mutex come attributo della stessa.\\
Il thread di invio è acquisce il mutex appena viene instaurata la connessione
e lo rilascia soltanto tramite la \emph{pthread\_cond\_timedwait}, ovvero
quando si mette in attesa di un evento, pertanto non è possibile che la 
variabile venga acceduta durante la gestione di uno degli eventi.
Per quanto riguarda la concorrenza tra il thread principale e il thread di
ricezione, il primo che accede alla variabile tramite l'acquisizione del
mutex segnala il proprio evento, poi se il secondo accede e trova che il tipo
di evento è diverso da NO\_EVENT aspetta fintanto che il thread di invio 
non se ne occupa e segnala di nuovo l'assenza di eventi tramite la funzione
\emph{pthread\_cond\_broadcast}.

\begin{lstlisting}[title=event.c]
int cond_event_signal(struct event *e, unsigned int event_type)
{
    int retval = 0;

    if (pthread_mutex_lock(&e->mtx) != 0)
        retval = -1;

    else {

        // wait until no event is signaled
        while (e->type != NO_EVENT)
            if (pthread_cond_wait(&e->cnd_no_event, &e->mtx) != 0)
                retval = -1;

        e->type = event_type;

        if (pthread_cond_signal(&e->cnd_event) != 0)
            retval = -1;

        if (pthread_mutex_unlock(&e->mtx) != 0)
            retval = -1;
    }

    return retval;
}
\end{lstlisting}
