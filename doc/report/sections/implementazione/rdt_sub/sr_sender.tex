\subsubsection{Selective Repeat: mittente}
Il lato mittente del protocollo viene svolto principalmente dal 
\emph{send\_service} e in parte anche dal \emph{recv\_service}.

Quando il lato applicativo intende inviare un messaggio, richiama la funzione
\emph{rdt\_send} passando dati e relativa quantità in byte come argomenti.\\
La funzione controlla lo spazio disponibile sul buffer circolare condiviso
e immette un quantità di dati (in byte) possibilmente pari ad un multiplo di 
MSS, per consentire al servizio di invio di creare segmenti completamente pieni 
di dati significativi (size = MSS). Se non sono disponibili almeno MSS byte 
aspetta fintanto che non si libera lo spazio necessario.\\
Essendo, il buffer circolare, una risorsa condivisa tra thread principale e 
thread di invio, è stato necessario implementare dei meccanismi di 
sincronizzazione.
A tale scopo l'accesso esclusivo al buffer viene garantito da un mutex,
mentre, per quanto riguarda l'attesa per la disponibilità di spazio, il thread
principale attende su una variabile condizione che viene opportunamente segnalata
quando il servizio di invio svuota il buffer, a sua volta il thread principale
segnala la condizione al thread di invio che nel buffer sono presenti dati dopo
l'immissione degli stessi.

\begin{lstlisting}[title=trasport.h]
struct circular_buffer {
    pthread_mutex_t mtx;
    pthread_cond_t cnd_not_empty;
    pthread_cond_t cnd_not_full;
    unsigned int S;
    unsigned int E;
    char buf[CBUF_SIZE];
};
\end{lstlisting}
\begin{lstlisting}[title=trasport.c]
void rdt_send(const void *buf, size_t len)
{
    size_t free, tosend, left = len;

    while (left) {

        if (pthread_mutex_lock(&send_cb.mtx) != 0)
            handle_error("pthread_mutex_lock");

        /* check available space */
        while ((free =
                space_available(send_cb.S, send_cb.E, CBUF_SIZE))
				<= MSS)
            if (pthread_cond_wait(&send_cb.cnd_not_full, 
				&send_cb.mtx) != 0)
                handle_error("pthread_cond_wait");

        /* calculate how much data to send */
        tosend =  free > left ?	left : free / MSS * MSS;

        memcpy_tocb(send_cb.buf, buf + len - left, tosend, 
					send_cb.E, CBUF_SIZE);

        send_cb.E = (send_cb.E + tosend) % CBUF_SIZE;

        if (pthread_mutex_unlock(&send_cb.mtx) != 0)
            handle_error("pthread_mutex_unlock");

        if (cond_event_signal(&e, PKT_EVENT) == -1)
            handle_error("cond_event_signal()");

        left -= tosend;
    }
}
\end{lstlisting}

Segnalata la presenza di dati sul buffer condiviso, il servizio di invio,
appena può, richiama la funzioni \emph{empty\_buffer} e \emph{send\_packets}.

\begin{lstlisting}[title=transport.c]
	.....

switch (e->type) {

case PKT_EVENT:

	/* empty shared buffer and put segments into the local one */
	empty_buffer(cb, pkts_buffer, &w, &lastseqnum);
	/* send available segments */
	send_packets(sockfd, loss, pkts_buffer, lastseqnum, &w,
				 &time_queue, &timeout);

	break;

case ACK_EVENT:
    // ack work .....
	break;

default:
	fputs("Unexpected event type\n", stderr);
	break;
}
	.....

\end{lstlisting}

La funzione \emph{empty\_buffer} ha il compito di svuotare il buffer condiviso,
creare i segmenti ed immagazzinarli nel buffer locale, in cui vi rimarranno
(saranno validi) fino a quando non sarà tornato indietro l'ack relativo e 
tutti quelli dei segmenti con numero di sequenza precedente.

\begin{lstlisting}[title=transport.c]
void empty_buffer(struct circular_buffer *cb, struct packet *pkts,
                  struct window *w, unsigned int *last_seqnum)
{
    size_t data, size;

    if (pthread_mutex_lock(&cb->mtx) != 0)
        handle_error("pthread_mutex_lock");

    while (cb->S != cb->E && 
           cbuf_free(w->base, *last_seqnum, MAXSEQNUM)) {
        // shared buffer not empty and local buffer has free slots

        data = data_available(cb->S, cb->E, CBUF_SIZE);
        size = data < MSS ? data : MSS;

        /* store a new packet */
        store_pkt(pkts, *last_seqnum, size, cb);
        *last_seqnum = (*last_seqnum + 1) % MAXSEQNUM;
        cb->S = (cb->S + size) % CBUF_SIZE;

        if (pthread_cond_signal(&cb->cnd_not_full) != 0)
            handle_error("pthread_cond_signal");
    }

    if (pthread_mutex_unlock(&cb->mtx) != 0)
        handle_error("pthread_mutex_unlock");
}
\end{lstlisting}
La funzione, fintanto che sono presenti dati sul buffer condiviso e ci sono 
slot liberi in quello locale, estrae al più MSS byte per poi creare un segmento,
che verrà posizionato in uno slot del buffer locale tramite la 
funzione \emph{store\_pkt}.\\
Ogni volta che un segmento viene estratto ed inserito nella rete viene anche 
aggiornato l'indice \emph{last\_seqnum} che indica il numero di sequenza del
prossimo pacchetto che verrà immagazzinato e quindi corrisponde al primo slot
libero del buffer locale.
\begin{lstlisting}[title=transport.c]
void store_pkt(struct packet *base, uint8_t seqnum, size_t size,
               struct circular_buffer *cb)
{
    struct packet *pkt = base + seqnum;
    struct segment *sgt = &(pkt->sgt);

    sgt->seqnum = seqnum;
    sgt->size = size;
    memcpy_fromcb(sgt->payload, cb->buf, size, cb->S, CBUF_SIZE);

    pkt->rtx = false;
}
\end{lstlisting}

La funzione \emph{send\_packets} si occupa di inviare i segmenti presenti nel
buffer locale fino a che l'indice nextseqnum, non superi il limite stabilito
dall'ampiezza della finestra di invio.
\begin{lstlisting}[title=transport.c]
void send_packets(int sockfd, double loss, struct packet *pkts,
                  unsigned int lastseqnum, struct window *w,
                  struct queue_t *time_queue, 
				  struct timespec *timeout)
{
    static unsigned int nextseqnum = 0;
    struct packet *pkt;         // packet pointer

    while (in_window(w, nextseqnum) &&
           more_packets(nextseqnum, w->base, lastseqnum)) {
        // nextseqnum is inside the window and
        // there are packets not sent yet

        pkt = pkts + nextseqnum;
        send_packet(sockfd, pkt, loss);

        /* set packet sendtime and exptime */
        pkt_settime(pkt, timeout);

        prio_enqueue(pkt, time_queue, pkt_exptimecmp);
        nextseqnum = (nextseqnum + 1) % MAXSEQNUM;
    }
}
\end{lstlisting}
Quando un segmento viene inviato vengono registrati il tempo di invio e di
scadenza, inoltre un riferimento al pacchetto viene inserito nella coda 
per la gestione del timeout.\\
Una volta che almeno un segmento è stato inviato, entra in gioco il 
\emph{recv\_service} che, bloccato in lettura sulla socket, attende l'arrivo
degli ack.\\
Quando ne giunge uno, appena possibile lo segnala al \emph{send\_service} 
inserendo il numero di sequenza nella variabile \emph{acknum} della
struttura condivisa \emph{event}.
\begin{lstlisting}[title=transport.c]
.....

for (;;) {

	r = read(sockfd, buffer, max_recvsize);

	.....

	/* segment received */
	if (r == sizeof(struct segment)) {
		// segment work .....
		continue;
	}

	/* ACK received */
	if (r == sizeof(acknum)) {
		acknum = (uint8_t) * buffer;
		if (cond_ack_event_signal(e, acknum) == -1)
			handle_error("cond_event_signal()");
		continue;
	}

.....
\end{lstlisting}

Alla ricezione del segnale di ack il servizio di invio aggiorna il valore 
del timeout e la finestra, eventualmente facendola scorrere se il numero
di sequenza è relativo alla base della finestra stessa.
\begin{lstlisting}[title=transport.c]
switch (e->type) {

    case PKT_EVENT:
        // pkt work .....
        break;

    case ACK_EVENT:
        acknum = e->acknum;
        if (params->adaptive)
            update_timeout(&timeout, pkts_buffer + acknum);
        update_window(&w, acknum);
        empty_buffer(cb, pkts_buffer, &w, &lastseqnum);
        send_packets(sockfd, loss, pkts_buffer, lastseqnum, &w,
                     &time_queue, &timeout);
        break;

    default:
        fputs("Unexpected event type\n", stderr);
        break;
    }
}
\end{lstlisting}
Il valore del timeout viene aggiornato soltanto quando arrivano ack relativi
a segmenti che non hanno subito ritrasmissioni.\\
La funzione \emph{adapt\_timeout} aggiorna il timeout in base a quanto tempo
è trascorso dall'invio del segmento fino alla ricezione dell'ack, come
descritto nel libro di testo [riferimento], secondo la formula [formula]

Il tempo trascorso è calcolato sottraendo all'istante corrente quello di invio
del segmento ricevuto.
\begin{lstlisting}[title=transport.c]
void update_timeout(struct timespec *timeout, struct packet *pkt)
{
    struct timespec now, elapsed;

    /* check if packet was retransmitted */
    if (pkt->rtx)
        return;

    if (clock_gettime(CLOCK_REALTIME, &now) == -1)
        handle_error("ack timestamp");

    if (timespec_sub(&elapsed, &now, &pkt->sendtime) == -1)
        handle_error("calculating elapsed time");

    adapt_timeout(timeout, &elapsed);
}
\end{lstlisting}
L'aggiornamento della finestra consiste nel contrassegnare il segmento come
ricevuto. Questo viene fatto impostando il relativo bit a 1 nella barra degli
ack della finestra.\\
Se il numero di sequenza corrisponde alla base della finestra, il bit non viene
impostato e viene fatta scorrere la finestra relativamente al prossimo segmento
per cui deve ancora arrivare un riscontro.
\begin{lstlisting}[title=window.c]
void update_window(struct window *w, uint8_t acknum)
{
    unsigned int i, s;

    if (acknum == w->base) {
        /* shift ack bar to the first unmarked bit */
        s = calc_shift(w);
        if (shift(&w->ack_bar, s) == -1)
            handle_error("shift()");
        /* slide window */
        w->base = (w->base + s) % MAXSEQNUM;
    }
    else if (in_window(w, acknum)) {
        /* calculate distance from window's base */
        i = distance(w, acknum);
        /* mark packet as acked */
        if (set_bit(&w->ack_bar, i) == -1)
            handle_error("set_bit()");
    }
}
\end{lstlisting}
Infine, se c'è stato uno scorrimento della finestra di invio, si libererà
dello spazio sul buffer circolare e sarà possibile 
inviare i segmenti successivi, pertanto vengono richiamate le funzioni
\emph{empty\_buffer} e \emph{send\_packets} già descritte in precedenza.

L'ultimo evento a cui il mittente deve reagire riguarda la scadenza del 
timeout di un segmento.\\
Ciò avviene quando la \emph{pthread\_cond\_timedwait} restituisce la 
costante ETIMEDOUT come valore di ritorno, in tal caso viene chiamata
la funzione \emph{resend\_expired} che determina quali pacchetti sono 
scaduti e li ritrasmette.
\begin{lstlisting}[title=window.c]
for (;;) {

    .....

	condret = 0;
	while (e->type == NO_EVENT && condret != ETIMEDOUT) {
		// no events and timeout not expired

		/* calculate remaining time to wait */
		if (calc_wait_time(&time_queue, &wait_time) == -1) {
			/* timeout expired: resend expired packets */
			resend_expired(sockfd, loss, &time_queue, &timeout, &w);
			continue;
		}

		condret = pthread_cond_timedwait(&e->cnd_event, &e->mtx,
                                         &wait_time);
		if (condret != 0 && condret != ETIMEDOUT)
			handle_error("pthread_cond_timedwait");
	}


	/* TIMEOUT EVENT */
	if (condret == ETIMEDOUT) {
		resend_expired(sockfd, loss, &time_queue, &timeout, &w);
		continue;
	}

    .....
}
\end{lstlisting}
Nello specifico la funzione per la ritrasmissione estrae segmenti fintanto
che non ne trova uno che non sia scaduto, per ognuno di essi si controlla 
se non è stato già ricevuto ed in tal caso il segmento viene inviato di
nuovo, marcato come ritrasmesso e vengono aggiornati i tempi di invio e di
scadenza, infine viene di reinserito nella coda.
\begin{lstlisting}[title=transport.c]
void resend_expired(int sockfd, double loss, 
                    struct queue_t *time_queue,
                    struct timespec *timeout, struct window *w)
{
    struct packet *pkt;

    while (time_queue->head != NULL) {

        /* fetch first to expire packet */
        pkt = get_head_packet(time_queue);

        if (!pkt_expired(pkt))
            break;

        dequeue(time_queue);

        /* check if packed has been acked */
        if (pkt_acked(w, pkt->sgt.seqnum))
            continue;

        send_packet(sockfd, pkt, loss);
        pkt->rtx = true;

        /* set packet time */
        pkt_settime(pkt, timeout);

        prio_enqueue(pkt, time_queue, pkt_exptimecmp);
    }
}
\end{lstlisting}
Un'operazione che può sembrare superflua è qualla relativa al controllo sulla 
ricezione del segmento, questo passaggio è necessario poiché quando si 
riceve un ack, non viene rimosso il nodo del relativo segmento, perché 
altrimenti si dovrebbe effettuare una ricerca nella coda del nodo da rimuovere
e questo costerebbe $\mathcal{O}(N)$ in termini di prestazioni, pertanto si è 
scelto di rimuovere il nodo nel momento in cui la coda viene scansionata per 
il controllo delle scadenze.\\
Tuttavia questa scelta, in particolari condizioni, potrebbe provocare una
situazione spiacevole.
Infatti se la probabilità di perdita è molto bassa ($ <10\%$) e la finestra 
di invio scorre troppo velocemente prima che i nodi relativi a pacchetti già 
riscontrati vengano rimossi, è possibile che i nl

Poiché ciò avviene solo in condizioni particolari, si è scelto di mantenere
l'implementazione con le rimozioni dei nodi ritardate, in modo da favorire le
prestazioni.

La funzione \emph{pthread\_cond\_timedwait} attende il verificarsi degli 
eventi di ricezione di dati dal livello superiore oppure di ricezione di
un ack, per un tempo calcolato ad ogni ciclo tramite la funzione
\emph{calc\_wait\_time} che imposta il tempo di attesa in base a quanto 
ne rimane alla scadenza del timeout del primo segmento.\\
Se la coda non contiene nodi, il timeout viene impostato ad un valore
molto elevato per fare in modo che il thread non venga svegliato 
inutilmente.
\begin{lstlisting}[title=transport.c]
int calc_wait_time(struct queue_t *q, struct timespec *wait_time)
{
    struct packet *pkt;
    struct timespec now, left;

    if (clock_gettime(CLOCK_REALTIME, &now) == -1)
        handle_error("clock_gettime()");

    if (!q->head) {
        /* queue is empty: turn off the timeout */
        left.tv_sec = 15;
        left.tv_nsec = 0;
    } else {
        pkt = get_head_packet(q);

        /* calculate remaining time to timeout */
        if (timespec_sub(&left, &pkt->exptime, &now) == -1)
            /* now > exptime: timeout expired */
            return -1;
    }

    timespec_add(wait_time, &now, &left);
    return 0;
}
\end{lstlisting}
