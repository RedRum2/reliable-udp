%
Il servizio di ricezione invece risponde ai seguenti eventi:
\begin{itemize}
\item[-]Ricezione di dati dalla rete (segmenti o ack);
\item[-]Scadenza di un timeout relativo alla connessione.
\end{itemize}
%
In caso di ricezione di dati dalla rete, segmenti e ack vengono distinti in 
base alla loro dimensione, invece il timeout è implementato impostandolo
sulla socket in lettura.
%
\begin{lstlisting}[title=transport.c]
void *recv_service(void *p)
{
	.....

    for (;;) {

        r = read(sockfd, buffer, max_recvsize);

        if (r == -1) {
            if (errno == EINTR)
                // signal interruption
                continue;
            if (errno == EAGAIN || errno == EWOULDBLOCK) {
                // timeout expired: close connection
                puts("Connection expired\n");
                exit(EXIT_SUCCESS);
            }
            handle_error("recv_service - read()");
        }

        /* segment received */
        if (r == sizeof(struct segment)) {

            // segment work .....
            
            continue;
        }

        /* ACK received */
        if (r == sizeof(acknum)) {

            // ack work .....

            continue;
        }

        fputs("recv_service: undefined data received\n", stderr);
    }

	.....
}
\end{lstlisting}
