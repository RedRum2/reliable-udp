\subsubsection{Segmenti}
Lo strato di trasporto virtuale è visto dall'esterno come una ``scatola nera''
che prende in ingresso messaggi dal livello applicativo, e restituisce i 
messaggi di risposta dell'host interlocutore.\\
Nello specifico un messaggio viene frammentato in segmenti di una misura 
massima prefissata (MSS), i quali poi vengono inviati e gestiti tramite 
l'algoritmo di trasferimento affidabile. Questa ulteriore frammentazione è 
necessaria per evitare che un segmento venga ulteriormente suddiviso a livello
di collegamento, cosa che potrebbe causare la perdita parziale dei segmenti.\\
La dimensione massima del segmento è stata calcolata considerando un MTU 
relativo ad un collegamento Ethernet standard di 1500 byte, un header UDP/IP
di 28 byte ed un header contente i parametri necessari all'esecuzione 
dell'algoritmo: il numero di sequenza del segmento pari ad 1 byte e la 
quantità di byte significativi nel payload pari a 2 byte.
Il numero di sequenza è contenuto in una variabile da 8 bit, pertanto può
variare da 0 a 255 (MAXSEQNUM - 1).
%MSS = MTU - UDP/IP - SR
\begin{lstlisting}[title=transport.h]

#define MAXSEQNUM       (1 << 8)
#define MTU             1500
#define UDPIP_HEADER    28
#define SR_HEADER       (sizeof(uint8_t) + sizeof(uint16_t))

#define MSS             (MTU - UDPIP_HEADER - SR_HEADER)


struct segment {
	uint8_t  seqnum;
	uint16_t size;
	uint8_t  payload[MSS];
};
\end{lstlisting}
